\chapter*{Introduction}
\addcontentsline{toc}{chapter}{Introduction}
Answer Set Programming (ASP) is a highly utilized framework for knowledge 
representation and automated reasoning, as highlighted by \textit{Marek and Truszczyński (1999)} ~\cite{MarekT99}
and \textit{Niemelä (1999)} ~\cite{Niemela99}. 
In ASP, combinatorial problems are formulated using logical rules that incorporate various
linguistic constructs, which simplify the representation of complex knowledge.
In its most basic form, ASP programs consist of normal logic rules, where each rule 
has a head atom and a body that is a conjunction of literals. 
Often, normal programs are extended by incorporating aggregates, as discussed by 
\textit{Bartholomew et al. (2011)} \cite{DBLP:conf/aaaiss/BartholomewLM11}, \textit{Faber et al. (2011b)}
\cite{DBLP:journals/ai/FaberPL11}, \textit{Ferraris (2011)} \cite{DBLP:journals/tocl/Ferraris11},
\textit{Gelfond and Zhang (2014)} \cite{DBLP:journals/tplp/GelfondZ14},
\textit{Liu et al. (2010)} \cite{DBLP:journals/ai/LiuPST10},\textit{ and Simons et al. (2002)}
\cite{DBLP:journals/ai/LiuPST10}.
Specifically, \textit{SUM aggregates} are used in rule 
bodies, where literals are assigned weights, 
and the sum of the weights of the true literals must satisfy a specified (in)equality.
When the head of the rule is false the aggregate becomes a constraint aggregate.
Another kind of constraint is the \textit{At Most One} constraint
that essentially inhibits truth of pairs of literals in a given set.
It is very common that these two constraints (SUM and AMO) are linked together.
State of the art solvers treat this case ignoring the correlation between these two constraints. 
Our work aims to define a new costruct named \textit{AMO-SUM} to efficiently handle this case.
This efficiency improvement comes from the fact that we are able to treat the two constraint
as a whole constraint.

% The solutions to these problems are given in the form of stable models, 
% as described by \textit{Gelfond and Lifschitz (1990)} ~\cite{GelfondL90}. 
% In its most basic form, ASP programs consist of normal rules, where each normal rule 
% has a head atom and a body that is a conjunction of literals. 
% Intuitively, the head atom must be true whenever the associated body is true, 
% making stable models classical models of the propositional knowledge base created by
% translating normal rules into implications (body implies head). 
% The stability condition also ensures that stable models have additional properties, 
% such as being supported models (where every true atom is the head of some rule with a true body)
% as noted by \textit{Fages (1994)} ~\cite{DBLP:journals/mlcs/Fages94}, and unfounded-free models
% (where the support is acyclic) as described by 
% \textit{Dung (1992)} ~\cite{DBLP:journals/tcs/Dung92}.

% These properties are leveraged to achieve efficient computation by 
% integrating a conflict-driven clause learning 

Nowadays current ASP solver implements a (CDCL) algorithm with propagators, 
as explained by \textit{Gebser et al. (2012)} ~\cite{DBLP:journals/ai/GebserKS12}.
CDCL (Conflict-Driven Clause Learning) is a contemporary form of non-chronological 
backtracking that follows the \textit{choose-propagate-learn} pattern, 
as described by \textit{Marques-Silva et al. (2021)} ~\cite{DBLP:series/faia/0001LM21}, ~\cite{DIMCAS}.

% The \textbf{choose} phase, or decision phase, consists in the process
% of \textit{choosing} a (branching) literal as the new literal to become true.
% After a literal has been choosed the so-called \textbf{propagate} phase takes place, whose role 
% is to extend the current partial assignment: a partial assigment
% is a non-total function on the atoms of the program that maps each of them to a boolean value.
% Extending means mapping literals to boolean value that can be deterministically inferred from the
% program and the current partial assignment. Specifically, process propagate employs a series of propagation functions,
% often referred as \textit{propagators}, following a priority sequence.
% The \textbf{learn} phase occurs when a conflict is detected, that is when the partial assignment contains a contradiction.
% In this stage a new clause has to be learned, to not make the same mistake again.
% To learn this clause each inferred literal is linked to a reason, which is a clause (a set of literals)
% that led to derived the literal in the propagation phase.
% When conflicts arise, these reasons are used to learn new clauses,
% effectively pruning the search space and enhancing the efficiency of the search process.
This pattern consists of this three phases: 
\textbf{choose} phase, or decision phase, consists in picking a branking literal as true; 
\textbf{propagate} phase derives determinists consequences of the current state;
\textbf{learn} phase is triggered when a conflict arises and aims at understanding from the conflict 
to not making the same mistake again.
In the propagate phase  specific procedures called \textit{propagators} are used to derive such consequences.
Propagators are required to explain why a consequence has been derived.
This explanation is called reason, it is a set of literals that led to derive that consequence 
and it is used in the learning phase.
When an aggregate is introduced inside the program then a specific propagator is required.
Our work provides both a novel propagator for handling the new AMO-SUM construct and 
an algorithm to minimize the reasons of the aggregate-derived consequences.
To provide a more comprehensive understanding of our work the \todo{DA FINIRE}




