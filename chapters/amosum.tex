\chapter{AMOSUM}
\label{cp:AMOSUM}
In this chapter we propose a new constraint with 
relative syntax and semantics (section ~\ref{sec:amo:syntax_semantics}).
This constraint combines a SUM constraint of the form  \eqref{eq:sum} 
with a collection of AMO constraints of the form \eqref{eq:amo}.
Then the related rules for propagating are described in section ~\ref{sec:amo:inference_rules}.
Finally the \textit{Propagate} and the \textit{Unroll} functions are defined 
in the last two sections

\section{Syntax and Semantics}
\label{sec:amo:syntax_semantics}
An \textit{AMOSUM} constraint is defined as follows:
\begin{align}\label{eq:amosum}
    \amosum\{w_1 : \ell_1\ [g_1];\ \cdots;\ w_n : \ell_n\ [g_n]\} \geq b
\end{align} 
where $n \geq 0$, $\ell_1,\ldots,\ell_n$ are distinct literals such that $\ell_i \neq \overline{\ell_j}$ (for all $1 \leq i < j \leq n$),
and $b,w_1,\ldots,w_n,g_1,\ldots,g_n$ are naturals numbers. 
As in \eqref{eq:sum}: $\{w_1,\ldots,w_n\}$ is the set of weights and $\mathit{wh}_{\sigma}(l_i) = w_i$;
$\mathit{bnd}_{\sigma} = b$ is the \textit{bound} of the constraint $\sigma$.
The new term is $g_i$, it represents the \textit{group id} of the literal; every literal with the same 
group id is inside an \textit{AMO} constraint.
Relation ${}\in{}$ is now defined as $(w_i : \ell_i\ [g_i]) \in \sigma$ for all $i \in [1..n]$.
$\mathbb{G}_{\sigma}$ is the set of possible group id, it means that $g_i \in \mathbb{G}_{\sigma}$ for all $1 \leq i \leq n$.
Given a literal $\ell_i$ in the aggregate,  $\mathit{group}_{\sigma}$ is function that maps $\ell_i$ to its group id, that is, $\mathit{group}(\ell_i) = g_i$.
Given a group id $g \in \mathbb{G}_ {\sigma}$ then all the literals in the same group are 
defined $\mathit{lits}_{\sigma}|_g = \{ \ell \mid (w : \ell [g]) \in \sigma, w \in \mathbb{N}\}$.
The relation $\models$ for a constraint $\sigma$ of the form \eqref{eq:amosum}, defined in ~\ref{sec:bg-syntax_semantics}, is
extended as follows: given an interpretation $I$, $I \models \sigma$ if $\sum_{i = 1}^{n}{w_i \cdot I^{\top}(\ell_i)} \geq \mathit{bnd}_\sigma$,
and $\sum_{\ell \in \mathit{lits}_{\sigma}{|_g}}{I^{\top}(\ell)} \leq 1$ for all $g \in \mathbb{G}_\sigma$.
If the subscript $\sigma$ is clear from context, it will be omitted.

The corresponding set of constraints defining an AMOSUM \eqref{eq:amosum} using just SUM constraints of the form \eqref{eq:sum}
is the following:
\begin{align*}
    & \textsc{sum}\{w_1 : \ell_1;\ \cdots;\ w_n : \ell_n\} \geq b\\
    & \textsc{sum}\{1 : \overline{\ell_{1}^g};\ \cdots;\ 1 : \overline{\ell_{m_{g}}^g}\} \geq m_{g} - 1 \quad g \in G
\end{align*}
where $l_{i}^g$ is the \textit{i-th} literal in the group $g$ and $m_{g}$ is the number of literals in the group $g$.

To provide a more concrete illustration,
a concrete example is given.
\begin{example}[Continuing Example~\ref{ex:running}]\label{ex:running-revised}
    $\Pi_{\mathit{run}}$ is rewritten by replacing $\sigma_1$ and $\sigma_2$ with
    \begin{align*}
    \begin{array}{rl}
        \sigma_{3}: & \amosum\{
            1 : x\ [1];\ 2 : y\ [1];\ 2 : z\ [2]
        \} \geq 3
    \end{array}
    \end{align*}
    Note that $\mathit{G}_{\sigma_3}=\{1, 2\}$, 
    $\mathit{group}_{\sigma_{3}}(x) = \mathit{group}_{\sigma_{3}}(y) = 1$, 
    $\mathit{group}_{\sigma_{3}}(z) = 2$, 
    $\mathit{lits}_{\sigma_{3}}|_1 = \{x, y\}$, and
    $\mathit{lits}_{\sigma_{3}}|_2 = \{z\}$.
\end{example}
\section{Inference rules}
\label{sec:amo:inference_rules}
The propagator for constraint ~\ref{eq:amosum} has 3 inference rules,
the first 2 have a counterpart in the classical setting, instead the last
one is a totally new one.

\paragraph{AMO inference rule} 
\label{eq:amosum:propagation:1}
The first inference rule is the one ensuring the at most one constraint 
~\eqref{eq:amo}: given a literal 
$\ell$ such that $(w : \ell [g]) \in \sigma$ for some $w \in \mathbb{N}$ and $g \in \mathbb{G}$,
then $\ell$ is inferred as false, i.e., $\overline{\ell} \in I$, if there exists 
$\ell' \in \mathit{lits}|_g$ such that $\ell' \in I$. In this case
$\mathit{reason}(\ell')=\{\overline{\ell},\overline{\ell'}\}$

\paragraph{SUM inference rule} 
This inference rule has a corresponding couterpart in the SUM constraint, that is,
a literal is inferred as true if it is required to reach the bound.
As it is done in \eqref{eq:sum:propagation} the concept of \textit{max possible sum} is used, a literal  $\ell$ is required to be true if all the literals 
in $\mathit{lits}|_{group(\ell)} \setminus \{ \ell \}$ are 
falses (i.e., it is the only not false literal in its group) and 
the \textit{maximum possible sum} (considering $\ell$ as false) would be less than $\mathit{bnd}$.
In this case the max possible sum is different, since not all the literals are free to contribute to the overall 
sum; just one literal per group can be true (i.e., contribute to the sum).
To get the the maximum possible sum it is enough to pick the maximum not false literal from each group;
more formally: $\mathit{mps\_amo}_{\sigma}(I,\ell) = \sum_{g \in \mathbb{G}_\sigma \setminus
\{\mathit{group}_\sigma(\ell)\}}{\mathit{mwh}_\sigma(I,g)}$ 
where $\mathit{mwh}_\sigma(I,g) := \max\{w \cdot I^{\neg\bot}(\ell) \mid (w : \ell\ [g]) \in \sigma\} \cup \{0\}$ 
is the maximum weight that group $g$ can contribute to the overall sum. 
Finally, the literal $\ell$ is added to $I$ if the following condition holds 
\begin{align}\label{eq:amosum:propagation:2}
    \mathit{mps\_amo}_{\sigma}(I,\ell) < \mathit{bnd}_\sigma
\end{align}
Furthermore, other functions are defined:\\$\mathit{mwh}_{\sigma}(g) = 
\max\{ w \mid (w : x [g]) \in \sigma \} \cup \{0\} $
;$\ml_{\sigma}(g) =  
\arg \max_{e \in S} \wh(e)$ 
where $S = \{x \mid (w : x [g]) \in \sigma, w \in \mathbb{N}\}$;
$\ml_{\sigma}(I,g) =  
\arg \max_{e \in S} \wh(e) \cdot I^{\neg\bot}(e)$

Henceforth, unless otherwise specified, \textit{mps} will be used in place of \textit{mps\_amo}.
In this case, $\mathit{reason}(\ell)$ is
\begin{align}\label{eq:amosum:reason:2}
\mathit{lits}_\sigma|_{\mathit{group}_\sigma(\ell)} \cup 
\bigcup_{g \in \mathbb{G}_\sigma \setminus \{\mathit{group}_\sigma(\ell)\}}{
    \mathit{just}_\sigma(I,g)
},
\end{align}
where 
$\mathit{just}_\sigma(I,g) := \{\overline{\ell'}\}$ if $\ell' \in \mathit{lits}_\sigma|_g \cap I$ 
(i.e., there exists a true literal in the group $g$), and
$\{\ell' \in \mathit{lits}_\sigma|_g \mid \mathit{wh}_\sigma(\ell') > \mathit{mwh}_\sigma(s)\}$ 
otherwise (i.e., the false literals in the group $g$ that could had increased the overall sum).

\paragraph{Enforced falsity rule}
The third inference rule has not counterpart in AMO or SUM constraints.
This rule enfore falsity of a literal that it is guaranteed to lead the max possible 
sum under the bound if that literal was true.
Given a literal $\overline{\ell}$ it is added to $I$ if the following condition holds:
\begin{align}\label{eq:amosum:propagation:3}
    \mathit{mps}_{\sigma}(I,\ell) + \mathit{wh}_\sigma(\ell) < \mathit{bnd}_\sigma
\end{align}
The rational behind this inference rules is the following: if $\ell$ was true it would 
contribute to the \textit{mps}, if with this hypothesis the mps would be less than the bound then 
it means that $\ell$ must be false.
In this case, $\mathit{reason}(\overline{\ell})$ is
\begin{align}\label{eq:amosum:reason:3}
\{\overline{\ell}\} \cup \bigcup_{g \in \mathbb{G}_\sigma \setminus \{\mathit{group}_\sigma(\ell)\}}{
    \mathit{just}_\sigma(I,g)
}.
\end{align}

\begin{example}[Continuing Example~\ref{ex:running-revised}]\label{ex:amosum:propagation:2}
    Already when $I$ is empty, $\sigma_3$ infers $z$.
    In fact, $z$ is the last undefined literal in part 2, and \eqref{eq:amosum:propagation:2} gives
    \begin{align*}
         \max\{1 \cdot [\,]^{\neg\bot}(x), 2 \cdot [\,]^{{\neg\bot}}(y)\} = \max\{1 \cdot 1, 2 \cdot 1,0\} = 2 < 3
    \end{align*}
From \eqref{eq:amosum:reason:2},
$\mathit{reason}(z) = \{z\}$.
\end{example}

\begin{example}\label{ex:amosum:propagation:more-2}
    Let us consider the following constraint:
    \begin{align*}
    \begin{array}{rl}
        \sigma_{4}: & \amosum\{
            1 : x\ [1]; 2 : y\ [1]; 2 : z\ [2]; 3 : w\ [2]
        \} \geq 3
    \end{array}
    \end{align*}
    If $I = [\overline{w}]$, the second inference rule associated with $\sigma_4$ infers $z$ with $\mathit{reason}(z) = \{z, w\}$.
    %
    The same holds if $I = [\overline{x}, \overline{w}]$, with the addition of $y$ with $\mathit{reason}(y) = \{y, x, w\}$;
    note that $w$ is included in $\mathit{reason}(y)$ because it could increase the overall sum.
    %
    On the other hand, if $I = [\overline{y}, \overline{w}]$, then $x$ and $z$ are inferred with 
    $\mathit{reason}(x) = \{x, y, w\}$ and
    $\mathit{reason}(z) = \{z, w, y\}$;
    note that $y$ is included in $r_1 = \mathit{reason}(z)$ because it could increase the overall sum. 
    It is easy to see that if $I = [\overline{w}]$
    from \eqref{eq:amosum:propagation:2} $z$ is inferred with $r_2 = \mathit{reason}(z) = \{z, w\}$;
    note $r_2 \subset r_1$. This case arises from the fact that if $y$ was undefined then 
    $z$ would be inferred anyway, more technically: the \textit{increment} to the possible 
    sum, given from removing $y$ from the reason, is not enough to increase the mps up to the bound, thus 
    the reason without $y$ is a valid reason.
    As discussed in section ~\ref{sec:bg-SM}, a smaller reason in size is preferred.
    In chapter ~\ref{cp:minimizing_reason} an algorithm to compute the \textit{minimal} and 
    \textit{cardinality minimal} reason is proposed.
    Finally, if $I = [x, \overline{y}, \overline{w}]$, then $z$ is inferred with $\mathit{reason}(z) = \{z, w, \overline{x}\}$;
    again, in this specific case $x$ could be ignored.
    \end{example}
    

\section{Propagate}
\label{sec:amo:propagator}

In this section, we discuss the details of the \textit{Propagate} function implemented in the AMOSUM propagator.
The \textit{Propagate} function consists of two phases: \textit{update\_phase} and \textit{propagate\_phase}.
The \textit{update\_phase} updates the internal states, specifically the global variable $\mps$ (\textit{Max Possible Sum}), 
of the propagator and determines if the \textit{propagate\_phase} is necessary. 
The \textit{propagate\_phase} is the actual propagator, which implements the inference 
rules and related logic, as described in Section~\ref{sec:amo:inference_rules}.
The \textit{Propagate} function is defined in Algorithm~\ref{alg:Propagate}.
Assume the input to the \textit{Propagate} function is the literal $\ell$.

\begin{algorithm}[H]\small
    \caption{Propagate}
    \label{alg:Propagate}
    \SetKwInOut{Input}{Input}
    \SetKwInOut{Output}{Output}
    \Input{A constraint $\sigma$,  an interpretation $I$, a literal $\ell \in I$}
    \Output{A list of propagated literals}
    \Begin{

        $next\_phase \gets update\_phase(\sigma,I,\ell)$\;
        
        $propagated\_lits \gets \emptyset$\;
        \If{$next\_phase = \top$}{
            $propagated\_lits \gets propagate\_phase(\sigma,I)$\;
        }
        
        \Return $propagated\_lits$\;
    }
\end{algorithm}

The update phase, is described in algorithm ~\ref{alg:update_phase}.
Abusing of notation the relation $\in$ defined in section ~\ref{sec:amo:syntax_semantics}
is extended, defining $\ell \in \sigma$ true if $(w : \ell [g]) \in \sigma$ for some 
weight $w \in \mathbb{N}$ and group $g \in \mathbb{G}_{\sigma}$.  
If $\ell \in \sigma$ it means that it means that $\ell$ is true and it is in the 
aggregate set of $\sigma$. 
If $\ell$ becomes true, accordingly to AMO inference rule, it will contribute to the $\mps$ (since all other literals
will be inferred as false). 
If $\ell$ was already contributing to the $\mps$ (it was the maximum undefined), then the $\mps$ will not change. 
In this case, no literal will be inferred, and the next phase will not start. 
However, if $\ell$ was not previously contributing to the $\mps$, the $\mps$ will change accordingly.
To update the $\mps$ when a literal becomes true, it is necessary to remove the contribution of the previous literal and add the contribution of $\ell$
, that is,
$\mathit{mps}_\sigma \gets \mathit{mps}_\sigma -\mwh_\sigma(I,g) + \wh_\sigma(\ell)$.
In this case $\mps$ has changed so the next phase is required.
If, instead, $\overline{\ell} \in \sigma$, it means that $\overline{\ell}$ is false and may no longer contribute 
to the $\mps$.
To check this, it is sufficient to check that it 
was contributing and there was no true literal in its group;
more precisely: $\wh_\sigma(\overline{\ell}) = \mwh_\sigma(I \setminus \{\ell\},g)$
and $\lits_\sigma|_{\group(\ell)} \cap I = \emptyset$.
In this case, the contribution of $\ell$ has to be removed, since it is false,
and the new contribution ($\mwh_\sigma(I,g)$) has to be added to $\mps$.
In this case $\mps$ has changed so the next phase is required.
When $\wh_\sigma(\overline{\ell}) = \mwh_\sigma(I \setminus \{\ell\}, g)$ is false,
the following can occur: there is no true literal in $\lits|_g$ and there is only 
one undefined literal in $\lits|_g$. In this case, according to \eqref{eq:amosum:propagation:2},
some literal could be inferred. Therefore, regardless of the $\mps$, the next phase is required.
In all the other cases the next phases is not required.

\begin{algorithm}[H]\small
    \caption{update\_phase}
    \label{alg:update_phase}
    \SetKwInOut{Input}{Input}
    \SetKwInOut{Output}{Output}
    \Input{An aggregate $\sigma$, an interpretation $I$, a literal $\ell \in I$.}
    \Output{Boolean to move to the next phase}
    \Begin{
        \If{$\ell \in \sigma$}{
            $g \gets \mathit{group}_\sigma(\ell)$\;
            \If{$\mwh_\sigma(I,g) = \wh_\sigma(\ell)$} {
                \Return{$\bot$}
            }
            $\mathit{mps}_\sigma \gets \mathit{mps}_\sigma -\mwh_\sigma(I,g) + \wh_\sigma(\ell)$\;
        }
        \ElseIf{$\overline{\ell} \in \sigma$} {
            $g \gets \mathit{group}_\sigma(\overline{\ell})$\;
            \If{$\wh_\sigma(\overline{\ell}) = \mwh_\sigma(I \setminus \{\ell\},g)$
            and $\lits_\sigma|_{\group(\ell)} \cap I = \emptyset$ }    {  
            $\mathit{mps}_\sigma \gets \mathit{mps}_\sigma + \mwh_\sigma(I,g) - \wh_\sigma(\ell)$\;        
            }
            \ElseIf{$|\lits_\sigma|_{\group_\sigma(\ell)} \setminus (I \cup \overline{I})| = 1$ and 
            $\lits_\sigma|_{\group(\ell)} \cap I = \emptyset$}
            {
                \Return{$\top$}
            }\Else{\Return{$\bot$}}    
        }
        \Else{
            \Return{$\bot$}
        }
        \Return{$\top$}
    }
\end{algorithm}

Now, let's proceed to the \textit{propagation} phase. 
To infer a literal, we only need to consider its group and the current $\mps$. 
Thus, Algorithm~\ref{alg:propagate_phase} iterates through all groups to derive new literals. 
The first two blocks implement the \textit{AMO inference rule} and the \textit{Enforced falsity rule}. 
The \textit{SUM inference rule} must start after the first two since some new literals can be inferred as false, 
affecting the number of false literals for that group. 
To keep track of these false literals, a set named \textit{falses} is necessary. 
After computing this set, the third block can compute the \textit{SUM inference rule}.
Note: If 
\[
\lits_\sigma|_g \setminus (I \cup \mathit{falses}) = \{\ell\}
\]
and
\[
\lits_\sigma|_g \cap I = \emptyset,
\]
then \(\ell = \ml(I,g)\). This is because the maximum undefined literal cannot be inferred as false. If it could have been inferred as false, it would have been inferred as such in a previous iteration and would not have remained the maximum undefined literal.
That is, in the SUM inference rule $\wh(\ell) = \mwh(I,g)$, 
thus, $\mps - \wh(\ell) = \mps(I,\ell)$ (as defined in ~\ref{eq:amosum:propagation:2}).

\begin{algorithm}[H]\small
    \caption{propagate\_phase}
    \label{alg:propagate_phase}
    \SetKwInOut{Input}{Input}
    \SetKwInOut{Output}{Output}
    \Input{A constraint $\sigma$,  an interpretation $I$}
    \Output{A set $S$ of pairs \textit{(literal,reason)},}
    \Begin{
        $S \gets \emptyset$\\
        $\mathit{falses} \gets \emptyset$\\
        \For{$g \in \mathbb{G}$}{

            \tcp{AMO inference rule}
            \If{$\lits_\sigma|_g \cap I = \{\ell\}$}{
                $S \gets S \cup {(\overline{\ell_i},\{\overline{\ell_i},\overline{\ell})\}} 
                \quad \forall \ell_i \in \lits_\sigma|_{g} \setminus (\{\ell\} \cup I \cup \overline{I})$\\
                \textbf{continue}
            }

            \tcp{Enforced falsity rule}
            \For{$\ell \in \lits_\sigma|_g$}{
                \If{$\ell \not\in I \cup \overline{I}$}{
                    \If{$mps_\sigma - \mwh_\sigma(I ,g) + \wh_\sigma(\ell) < \mathit{bnd}_\sigma$}{
                        $\mathit{rns}_\ell = \lits_\sigma|_g \cup 
                        \bigcup_{x \in \mathbb{G}_\sigma \setminus \{g\}}{
                            \mathit{just}_\sigma(I,x)
                        }$\\
                        $S \gets (\overline{\ell},\mathit{reason}(\ell))$\\
                        $\mathit{falses} \gets \mathit{falses} \cup \{\ell\}$
                    }
                }
            }
            
            \tcp{SUM inference rule}
            \If{$\lits_\sigma|_g \setminus (I \cup \mathit{falses}) = \{\ell\}$ and $\lits_\sigma|_g \cap I = \emptyset$}{
                \If{$mps_\sigma - \wh_\sigma(\ell) < \mathit{bnd}_\sigma$}{
                    $S \gets S \cup (\ell,\{\overline{\ell}\} \cup \bigcup_{x \in \mathbb{G}_\sigma \setminus \{g\}}{
                        \mathit{just}_\sigma(I,x)
                    }) $
                }
            }
        }
    
        \Return $S$\;
    }
\end{algorithm}

\section{Unroll}
In addition to the \textit{Propagate} function, called when a literal has been chosen as a branching literal,
also a \textit{Unroll} function has to be defined, to update the internal state (\textit{mps}) of the propagator when a literal 
becomes undefined (due to some backjump in the search process).
Algorithm ~\ref{alg:Unroll} implements such procedure.
Given a literal $\ell$ that becomes \textit{undefined}, that is $\ell \not\in (I \cup \overline{I})$,
to update the $\mps$ it is necessary to know its previous value.
To have such information a \textit{boolean} variable $v$ is provided in input.
If $v = \top$ then $\ell$ was \textit{true}, otherwise it was \textit{false}.
If $\overline{\ell} \in \sigma$, to \textit{`simplify'} the algorithm, we only need to flip $v$ and $\ell$.
On one hand, if $\ell$ was true can happend that there is an undefined literal
with weight greater then $\ell$, i.e., $\mwh(I,g) > \wh(\ell)$;
in this case $\ell$ will not contribute anymore and 
the $\mps$ will increase of $\mwh(I,g) - \wh(\ell)$.
On the other hand, if $\ell$ was false, it may become the new maximum undefined,
and there might be no true literal in $\lits|_g$. 
Formally, if $\wh(\ell) > \mwh(I \cup \{\overline{\ell}\}, g)$ and $\lits|_g \cap I = \emptyset$, 
then $\ell$ will contribute to the $\mps$,
increasing it by $\wh(\ell) - \mwh(I \cup \{\overline{\ell}\}, g)$.

\begin{algorithm}[H]\small
    \caption{Unroll}
    \label{alg:Unroll}
    \SetKwInOut{Input}{Input}
    \SetKwInOut{Output}{Output}
    \Input{A constraint $\sigma$, an interpretation $I$, a literal $\ell \not\in (I \cup \overline{I})$, previous value $v$}
    \Output{Updated mps}
    \Begin{
        \If{$\ell \in \sigma$ or $\overline{\ell} \in \sigma $}{
            \If{$\overline{\ell} \in \sigma $}{
                $\ell \gets \overline{\ell}$\\
                $v \gets \neg v$\\
                
            }
            $g \gets \group(\ell)$\\
            \If{$v = \top$}{
                \tcp{$\ell$ was true}
                \If{$\mwh(I,g) > \wh(\ell)$}{
                    $mps \gets mps - \wh(\ell) +\mwh(I,g)$\;
                }
            }
            \ElseIf{$\wh(\ell) > \mwh(I \cup \{\overline{\ell}\},g)$ and $\lits|_g \cap I = \emptyset$}{
                \tcp{$\ell$ was false}           
                    $mps \gets mps - \mwh(I \cup \{\overline{\ell}\},g) + \wh(\ell)$\;
                }
            }
    }
\end{algorithm}