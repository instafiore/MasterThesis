\chapter{Minimizing reason}
\label{cp:minimizing_reason}
In this section, we address the problem of minimizing a reason for an AMOSUM 
constraint of the \(\ge\) type. In Section~\ref{sec:properties_reason},
we  discuss some important properties of the reason. Subsequently,
we introduce the concept of a \textit{redundant literal}
and explain the notion of \textit{increment},in section ~\ref{sec:redundant_literal} and ~\ref{sec:increment} respectively.
Utilizing these concepts,
we define two algorithms in the final sections: one for obtaining 
the minimal reason and the other for obtaining the cardinality minimal reason.

All the discussion applies on the inference rule ~\eqref{eq:amosum:propagation:2} for a constraint $\sigma$ of the from ~\eqref{eq:amosum}, thus, when it is possible, the subscript 
${\sigma}$ is omitted .


\section{Properties of a reason}
\label{sec:properties_reason}
An important property of a reason $R$ is that $\Pi \models R$, i.e., 
it is redundant.
Let $R_1,R_2$ be two reason of $z$ of the form:
\begin{align}
    \label{eq:reasons_1}
    &R_1 = \{z\} \cup \{\ell_1, \hdots, \ell_n, y\} \\ 
    \label{eq:reasons_2}
    &R_2 = \{z\} \cup \{\ell_1, \hdots, \ell_n, \overline{y}\}, 
\end{align}
where $n \ge 0$.
In this case $R_1 \setminus \{y\} = R_2 \setminus \{\overline{y}\}$ and $z,y \in R_1$ and 
$z,\overline{y} \in R_2$.
Let $\Pi$ be a program such that $\Pi \models R_1$ and $\Pi \models R_2$.
Let $I \models \Pi$ then $I \models R_1$ and $I \models R_2$, hence $I \models R_1 \land R_2$.
We have seen in ~\ref{sec:bg-SM} that 
$R_1 \land R_2 \rightarrow R_1 \otimes_{y} R_2$, thus $I \models  R_1 \otimes_{\ell} R_2$,
hence $\Pi \models R_1 \otimes_{y} R_2 = \{z\} \cup \{\ell_1, \hdots, \ell_n\} = R$.
Since $z \in R$ and $\Pi \models R$ then $R$ is a reason of $z$ in $\Pi$.
\begin{example} continuing example ~\ref{ex:amosum:propagation:more-2}\\
    When $I = [\overline{y}, \overline{w}]$ then $z$ is inferred,
    thanks to \eqref{eq:amosum:propagation:2},with $\mathit{reason}(z) = \{z,y,w\} = R_1$.
    If $I = [y, \overline{w}]$ then $mps(I,z) = \mathit{wh}(y) = 2 < 3$ and 
    $lits\mid_{group(z)} \setminus \overline{I} = \{z\}$ so $z$ is inferred 
    with $\mathit{reason}(z) = \{z,\overline{y},w\} = R_2$.
    Since $R_1 = \{z\} \cup \{w,y\}$,
    $R_2 = \{z\} \cup \{w,\overline{y}\}$
    are two reason of the form ~\ref{eq:reasons} then 
    $\Pi \models R_1 \otimes_{y} R_2 = \{z,w\} = R$ and $R$ is a reason of $z$.
\end{example}
We can see in the above example that $y$ can be removed from the reason, getting $R$,
and $R$ continues to be a reason, so it is \textit{redundant}.

\section{Redundant literal}
\label{sec:redundant_literal}
Let $L$ be set such that $\ell \in L$,
then $L_{\overline{\ell}} = (L \setminus \{\ell\}) \cup \{\overline{\ell}\}$.
A literal $\ell$ is \textit{redundant} in a reason $R$ of the literal $z$, under intepretation $I$, 
if $R_{\overline{\ell}}$ is a reason of $z$ under $I_{\ell}$.
If $\ell \in R$ then  $R,R_{\overline{\ell}}$ are of the form ~\ref{eq:reasons_1} and 
~\ref{eq:reasons_2} respectively,
and $R \otimes_{\ell} R_{\overline{\ell}} = R \setminus \{\ell\} = R'$ is a reason of $z$.
Note that $R'$ is a reason under  $I' = \overline{R' \setminus \{z\}}$; 
thus, given that $I' \subseteq I$, then $R'$ is a reason under $I$.
It is easy to see that every redundat literal of a reason $R$ can be removed from it.
\begin{example}{Continuing example ~\ref{ex:amosum:propagation:more-2}}\\
    \label{ex:redundant_lit}
    When $I = [\overline{y}, \overline{w}]$ then $z$ is inferred,
    thanks to \eqref{eq:amosum:propagation:2},with $\mathit{reason}(z) = \{z,y,w\} = R$;
    $R$ is a reason of $z$ under $I$.\\
    Let's check if $y$ is redundant.
    $R_{\overline{y}} = (R \setminus \{y\}) \cup \{\overline{y}\} = \{z,\overline{y},w\}$ and $I_{y} = [y,\overline{w}]$.
    With $J_{R_{\overline{y}}} = [y,\overline{w}]$ the propagator, thanks to \eqref{eq:amosum:propagation:2}, 
    would infer $z$.
    So, $R_{\overline{y}}$ is a reason for $z$
    and $y$ is redundant in $R$.
    Now, let's take the case where $I = [x, \overline{y}, \overline{w}]$, 
    and $z$ is inferred with $\mathit{reason}(z) = \{z, w, \overline{x}\} = R$, that is, 
    $\overline{w} \land  x \rightarrow z$.
    Let's check if $\overline{x}$ is redundant.\\
    $R_{x} = \{z, w, x\}$ and it is a reason of $z$ under $I_{\overline{x}}[\overline{w}, \overline{y},\overline{x}]$, 
    since $J_{R_x} = \overline{\{w, x\}} = [\overline{w},\overline{x}] \subseteq I_{\overline{x}}$
    infers $z$. Hence, $x$ is redundant in $R$.
\end{example}
Continuing the discussion about the example ~\ref{ex:redundant_lit}.
Let's examine the first case more in detail: $\mathit{reason}(z) = \{z, w, y\}$ is a 
reason of $z$ under $I = [\overline{y}, \overline{z}]$. This is true because,
given $J_R = \overline{\{z, w, y\} \setminus \{z\}} = \{w,y\} $,
$mps(J_R,z) = \mathit{wh}(x) = 1 < 3$ and $lits\mid_{group(z)} \setminus \overline{I} = \{z\}$.
Since $R_{\overline{y}} = \{z, w, \overline{y}\}$ then $J_{R_{\overline{y}}} = [\overline{w},y]$ and 
$mps(J_{R_{\overline{y}}},z) = \mathit{wh}(y) = 2 < 3$ and
$lits\mid_{group(z)} \setminus \overline{I_y} = \{z\}$, thus $z$ would be added in $I_y$.
More in concrete, $y$ is redundant because $mps(J_R,z) + \mathit{inc}(y) = 1 + 1 < \mathit{bnd} = 3$ and
$y \not\in group(z)$, where the $\mathit{inc}(y) = mps(J_{R_{\overline{y}}},z) - mps(J_R,z) = 2 - 1$.
Now, let's take the other case, where $I = [x, \overline{y}, \overline{w}]$, 
and $z$ is inferred with $\mathit{reason}(z) = \{z, w, \overline{x}\}$, that is, 
$\overline{w} \land  x \rightarrow z$.
As already seen $J_{R_{x}}=[\overline{x}, \overline{w}]$.\\
$mps(J_{R_{x}},z) = \mathit{wh}(y) = 2 < 3$ and
$lits\mid_{group(z)} \setminus \overline{I_{\overline{x}}} = \{z\}$, 
so $\mathit{reason}(z)=\{z, x, w\}$
and $\overline{x}$ is redundant.

\section{Increment}
\label{sec:increment}
As mentioned before, increment of $\ell$, written $\mathit{inc}(\ell)$, defines 
how much the $mps$ increases when that literal is flipped in the reason (to possibly
be removed).

\subsection{False literal}
Let $\ell$ be a \textit{false} literal that is,
$\overline{\ell} \in I$. 
A reason $R$ of a literal $z$ under $I$ can contain $\ell$, let's assume $R$ to be 
such reason.
Let $J_R = \overline{R \setminus \{z\}}$, 
$J_{R_{\overline{\ell}}} = \overline{R_{\overline{\ell}} \setminus \{z\}}$, thus
$\ell \in J_{R_{\overline{\ell}}}$.\\
Note that $mps(J_R,z) = \sum_{g \in \mathbb{G} \setminus
\{\mathit{group}(z)\}}{\mathit{mwh}(J_R,g)}$ and \\
$mps(J_{R_{\overline{\ell}}},z) = \sum_{g \in \mathbb{G} \setminus
\{\mathit{group}(z)\}}
{\mathit{mwh}(J_{R_{\overline{\ell}}},g)}$.
Since $J_R \setminus \lits|_{\group(\ell)} = J_{R_{\overline{\ell}}} \setminus \lits|_{\group(\ell)}$,
that is, all the literals outside $\group(\ell)$ remain with the same value;
then just the maximum weight of $\lits|_{\group(\ell)}$ can change.
Hence,
$$ \inc(\ell) = \mathit{mps}(J_{R_{\overline{\ell}}},z)-\mathit{mps}(J_r,z) = 
\mathit{mwh}(J_{R_{\overline{\ell}}},\mathit{group}(\ell)) - \mathit{mwh}(J_r,\mathit{group}(\ell))$$
Hence, when a literal has to be checked to be redundant and it is false then 
$\inc(\ell)$ has to be considered has the increment of the $\mps$.

\subsection{True literal}
Let $\ell$ be a \textit{true} literal that is,
$\ell \in I$. 
A reason $R$ of a literal $z$ under $I$ can contain $\overline{\ell}$, let's assume $R$ to be 
such reason.
Then  $J_R = \overline{R \setminus \{z\}}$ and $J_{R_{\ell}} = \overline{R_{\ell} \setminus \{z\}}$, thus
$\overline{\ell} \in J_{R_{\ell}}$.\\ 
Since $\ell \in I$ then $R$, by construction, does not containt any literal of $group(\ell)$ 
except $\ell$, then $R_{\ell}$ will do the same.
Since $J_{R_{\ell}} = \overline{R_{\ell} \setminus \{z\}}$ then also $J_{R_{\ell}}$ 
has no literal of group of $\ell$
except $\overline{\ell}$ (since it is false, it does not contribute to the $\mps$).
That is, $\mathit{mwh}(J_{R_{\ell}},group(\ell))= \max\{ \mathit{mwh}(x) \mid \mathit{group}(x) = group(\ell) \} 
= \mathit{mwh}(group(\ell))$.
It is easy to see that
$$\inc(\ell) = \mathit{mps}(J_{R_{\ell}},z)-\mathit{mps}(J_r,z) = \mathit{mwh}(group(\ell)) - \mathit{wh}(\ell)$$
Hence, when a literal has to be checked to be redundant and it is true then 
$\inc(\ell)$ has to be considered has the increment of the $\mps$.

\subsection{Algorithm}
In this section is proposed the algorithm computing the increment and its lower bound 0 is proved.
The increment is always non-negative since removing a literal from a reason 
in the worst case does not affect the mps.

\begin{theorem}
    \label{theorem:inc}
    Given a reason $R$ and a literal $\ell \in R$ then $\inc(\ell) \ge 0$.
\end{theorem}

\begin{proof}
    Case when $I \models \ell \in R$.
    Then $\inc(\ell) = \mwh(\group(\ell))-\mathit{wh}(\ell)$. Since $\mwh(\group(\ell))$ is the  
    maximum weight in $\lits \mid_{\group(\ell)}$ and $\ell \in \lits \mid_{\group(\ell)}$ then 
    $\mwh(\group(\ell)) \ge \mathit{wh}(\ell)$ and $\inc(\ell) \ge 0$.

    Case when $I \models \overline{\ell} \in R$.
    For readability let define $g = \group(\ell)$.\\
    Then $\inc(\ell) = \mathit{mwh}(J_{R_{\overline{\ell}}},g) - \mathit{mwh}(J_R,g).$\\
    Since $J_{R_{\overline{\ell}}} = (J_R \setminus \{\overline{\ell}\}) \cup \{\ell\}$ then 
    $$\mathit{mwh}(J_{R_{\overline{\ell}}},g) = 
    \max\{w \cdot J_R^{\neg\bot}(\ell) \mid (w : \ell\ [g]) \in \sigma\} \cup \{\wh(\ell)\}$$
    It is trivial to see that 
    $$\max\{w \cdot J_R^{\neg\bot}(\ell) \mid (w : \ell\ [g]) \in \sigma\} \cup \{\wh(\ell)\}
    \ge \max\{w \cdot J_R^{\neg\bot}(\ell) \mid (w : \ell\ [g]) \in \sigma\}$$
    Since $\max\{w \cdot J_R^{\neg\bot}(\ell) \mid (w : \ell\ [g]) \in \sigma\} = \mwh(J_R,g)$
    then $$\max\{w \cdot J_R^{\neg\bot}(\ell) \mid (w : \ell\ [g]) \in \sigma\} \cup \{\wh(\ell)\}
    \ge \mwh(J_R, g).$$
    Given that  $\max\{w \cdot J_R^{\neg\bot}(\ell) \mid (w : \ell\ [g]) \in \sigma\} \cup \{\wh(\ell)\} = 
    \mathit{mwh}(J_{R_{\overline{\ell}}},g)$, finally, 
    $$\mathit{mwh}(J_{R_{\overline{\ell}}},g) \ge \mwh(J_R, g)$$

    In both cases $\inc(\ell) \ge 0$.\\ Hence, $\inc(\ell) \ge 0$ for all $\ell \in R$
\end{proof}

The following algorithm computes the increment as previously described.
\begin{algorithm}[h]\small
    \caption{inc}
        \label{alg:cdcl}
        \SetKwInOut{Input}{Input}
        \Input{literal $\ell \in R$, interpretation $I$, reason $R$}.
        \SetKwBlock{Loop}{Loop}{end}
        \Begin{
            \If {$I \models \ell$}{
                \Return{ $\mwh(\group(\ell))-\mathit{wh}(\ell)$}
            }\Else{
                $J_R = \overline{R \setminus \{z\}}$ \\
                $J_{R_{\overline{\ell}}} = \overline{R_{\overline{\ell}} \setminus \{z\}}$\\
                \Return{$\mathit{mwh}(J_{R_{\overline{\ell}}},\mathit{group}(\ell)) - \mathit{mwh}(J_R,\mathit{group}(\ell))$}
            }
        }
\end{algorithm}

Given a reason $R$, intepretation $I$ and a set $S \subseteq S$
the \textit{total increment of S} denoted $\inc(S,I,R) = \sum_{\ell \in S} \inc(\ell, I, R \setminus S \cup \{\ell\})$ 

\section{Minimality}
In this section we  propose two algorithms to minimize the reason 
generated by the propagator. The first algorithm  (~\ref{subsec:minimal_reason}) is an algorithm
to get the minimal reason and it has linear complexity in the size of the reason.
In subsection ~\ref{subsec:cardinality_minimal_reason}), instead,
a new algorithm to compute the cardinality minimal reason is proposed,
with \textit{pseudo-polynomial} complexity.
For both algorithms a proof of correcntess is provided 
and for the second one \textit{NP-Hardness} is proved.

\subsection{Minimal reason}
\label{subsec:minimal_reason}

Given a reason $R$ of a literal $\ell$ then it is minimal if there  not exists 
a literal $\ell' \in R$ such that $R \setminus \{\ell'\}$ is a reason of 
$\ell$. In other words, $R$ is minimal if it does not containts redundant literals.
As seen before with each literal $\ell$ in a reason $R$ has an \textit{increment}.
That is, the increment to the mps when  $\ell$ is removed from the reason.

At this point, an important parallel must be drawn: 
minimizing a reason \(R\) equates to maximizing the
number of literals removed from \(R\). 
This first algorithm identifies the \textit{maximal} 
set of literals to be removed from \(R\).

\paragraph{Maximal Subset Sum}
As mentioned before finding the minimal reason is 
equal to find the maximal set of literal to remove from the reason.
Let $R$ be a reason of a literal $z$ under interpretation $I$ and
$S \subseteq R'$ be a set of literals where $R' = R\setminus\lits|_{\group(z)}$.
$S$ can be removed from $R$ if 
$$\sum_{e \in S}\inc(e,I,(R'\setminus S) \cup \{e\}) \le s $$
where $s = \mathit{bnd} - \mps(I,z) - 1$.
To comprehend the reasoning behind $s$, it is crucial to observe that, 
given $J = B(R'\setminus S)$ 
then $$\mps(J,z) = \mps(I,z) + \sum_{e \in S}\inc(e,I,(R'\setminus S) \cup \{e\})$$
For $R'\setminus S$ to remain a reason
the condition $\mps(R'\setminus S,z) \le \mathit{bnd} - 1$
must continue to hold.
Since $\mps(I,z) \le \mathit{bnd} - 1 $ (otherwise $R$ would not be a reason)
and $\sum_{e \in S}\inc(e,I,(R'\setminus S) \cup \{e\}) \le \mathit{bnd} - \mps(I,z) - 1$
then $$\mps(I,z) + 
\sum_{e \in S}\inc(e,I,(R'\setminus S) \cup \{e\}) \le \mps(I,z) + 
\mathit{bnd} - \mps(I,z) - 1 = \mathit{bnd} - 1$$.
Now, the entire algorithm can be defined. 
It begins with an empty set $S$ and a current increment $ci$
equal to 0. For each literal $\ell$ that is not in the group 
of $z$ it verifies if $\ell$ is \textit{redundant}.
If it is, it is added to the set $S$ and the $ci$
increase of $\inc(\ell)$. 
\begin{algorithm}[h]\small
    \caption{Maximal Subset sum (mss)}
    \label{alg:maximal_subset}
    \SetKwInOut{Input}{Input}
    \SetKwInOut{Parameters}{Paramenters}
    \SetKwInOut{Output}{Output}
    \Input{intepretation $I$, reason $R$ of $z$, intepretation $I$, threshold $s$}
    \Parameters{\textit{inc} function}
    \Output{subset maximal}
    \Begin{
        $\mathit{S} \gets [\,]$ \\
        $\mathit{ci} \gets 0$ \\
        \For{$\ell \in R \setminus \lits|_{\group(z)}$}{
            \If{$\mathit{ci} + \inc(\ell,I,R \setminus \mathit{S}) \leq s$}{
                $\mathit{ci} \gets$ $ci + \inc(\ell,I,R\setminus \mathit{S})$ \\
                $\mathit{S} \gets \mathit{S} \cup \{\ell\} $\\
            }
        }
        \Return{\textit{S}}
    }
\end{algorithm}

\subparagraph{Proof Correctness}

Before proving the theorem about the correcntess of algorithm ~\ref{alg:maximal_subset}
the following lemma has to be proved.

\begin{lemma}
    \label{lemma:maximal_subset}
    Given a reason $R$, a set $S \subseteq R$, a literal $\ell \in R \setminus S$
    and $\mathit{ci},s \in \mathbb{N}$.
    If $\mathit{ci} +  \inc(\ell,I,R \setminus \mathit{S}) > s$
    then every superset  $S' \supseteq S \cup \{\ell\}$ yields to 
    an increment greater of $s$
\end{lemma}

\begin{proof}
    The total increment of $S' = \inc(S',I,R) = \inc(S,I,R) + \inc(\ell,I,R \setminus \mathit{S}) + 
    \sum_{\ell \in S'} \inc(x, I, R \setminus S' \cup \{x\})$.
    Since $\inc(S,I,R) + \inc(\ell,I,R \setminus \mathit{S}) > s$ by assumption and 
    ,thanks to theorem ~\ref{theorem:inc}, $\inc(x, I, R \setminus S' \cup \{x\}) \ge 0$ then 
    $\inc(S',I,R) > s$.
    Hence, $S'$ is not a subset of $R$ giving as sum a value less than $s$.

\end{proof}

\begin{theorem}
    The Maximal Subset Sum algorithm returns the maximal subset $S$ where its sum 
    of increments of $S$ is less than $\mathit{bnd} - 1$
\end{theorem}

\begin{proof}
    Let $S_m$ the final subset returned by \textit{mss}.
    Arguing by contradiction, if it is not maximal 
    it means that there is a literal $\ell \in R \setminus S_m$ that could be
    added to $S_m$.
    But since $\mathit{ci} +  \inc(\ell,I,R \setminus \mathit{S}) \ge s$
    for some $S$.
    Since $S \subseteq S_m$, then $S \cup \{\ell\} \subseteq S_m \cup \{\ell\}$,;
    thanks to theorem ~\ref{lemma:maximal_subset} $S_m \cup \{\ell\}$
    is not a subset giving as sum a value less then $s$, that is, $\ell$ cannot be 
    added to $S_m$.
    Hence, $S_m$ is maximal,
\end{proof}


\subsection{Cardinality minimal reason}
\label{subsec:cardinality_minimal_reason}

This section, as previously mentioned, we propose a new algorithm to minimize the reason,
the main difference is that with algorithm ~\ref{alg:maximal_subset} the final reason 
is not guaranteed to be cardinality-minimal.
This new approach can find a cardinality-minimal reason, at the price of 
having a \textit{pseuso-polynomial} algorithm.
Before introducing the algorithm some preliminars have to be done.

\paragraph*{Preliminars} All the previous notation continue to hold.
Let $I$ an interpretation, $R$ be a reason of $z$ generated under $I$ 
and $S \subseteq R$ a set of redundant literals of $R$.\\
Function $\mathit{\ml_\inc} : \mathcal{P}(R) \times \mathit{G} \mapsto R$ 
returns the \textit{maximum increment literal in S of group g};
that is, 
$\ml_\inc(S,g)= \arg\max_{k \in S \cap \lits|_{{\group(g)}}} \inc(k)$.\\   
Given a set , a literal $\ell$ is active in $S$ if $\ml_\inc(S,\group(\ell)) = \ell$;
the set of active literal of $S$ is equal to $A_S = \{\ell \in S \mid \ell \text{ is active in S}\}$.
Given a literal $l \in R$ then $blw: R \mapsto \mathcal{P}(R)$ is a function 
that returns all the literals \textit{below} $\ell$, 
that is, $\blw(\ell) = \{ k \in \lits|_{\group(\ell)} \mid \inc(k,I,R) \le \inc(l,I,R) \}$.
Function $\mathit{sum}(S) = \sum_{\ell \in A_S} \inc(\ell,I,R)$ is the sum of all increments  
of active literals of $S$.\\
A extension of a set $S$ with a literal $\ell \in R \setminus S$, written as $S' = S \ext \ell$,
is equal to $S' = S \cup blw(\ell)$.\\
A sufficient condition for a literal $\ell$ 
to be active in $S' = S \ext \ell$ is that 
$S \cap \group(\ell) = \emptyset$.
Let $(\lits\_\mathit{ord}_g, \preceq)$ be an ordered set where $\lits\_\mathit{ord}_g = \lits|_g$
and $\preceq = \{(l,l') \in \lits|_g \times \lits|_g \,:\, \inc(l,I,R) \ge \inc(l',I,R)\}$.\\
Let $L = \{l^1_1,\hdots,l^1_{m_1},\hdots,l^g_1,\hdots,l^g_{m_g},\hdots,l^k_1,\hdots,l^k_{m_k} \}$
where $k = |\G|$, $m_g$ is the size of $lits\_\mathit{ord}|_g$ , $l^g_j$ is the j-th literal in $\lits\_\mathit{ord}_{g}$.
The set $L_j$ represent the first $j$ elements of $L$ and $n = |L|$.
Abusing of notation, given a literal $\ell_j \in L$, \{$\ell_j\} = L_j \setminus L_{j-1}$, i.e., $\ell_j$
is the \textit{j-th} literal in $L$.
% Let $M$ be a matrix where each cell is a subset of $L$ or it is equal to $\Box$.
% $M_{i,j} \subseteq L$  is named \textit{correct} 
% if it is the cardinality-maximal set with $sum(M_{i,j}) = i$ and $M_{i,j} \subseteq L_j$.
% If $M_{i,j} = \Box$ then it means that there are no subset of $L$ such that the total increment 
% is less or equal to $s$.

\paragraph{Cardinality-Maximal Set}
As done in section ~\ref{subsec:minimal_reason} to find a minimal reason $R'$ starting 
from $R$ we compute a maximal subset $S$ of $R$ of redudant literals to remove from $R$.
In the first approach, the cardinal-minimality property is not ensured, instead 
in this case $S$ is \textit{cardinality-maximal}.
Differently from the ~\ref{alg:maximal_subset}, here the increment is \textbf{always}
with respect the \textit{`initial'} increment, that is, with respect to the initial reason $R$;
instead in the first algorithm the increment dependes from the \textit{current} reason.
  

The algorithm ~\ref{alg:cmss} computes the cardinality-maximal subset $S$ of $L$ such that 
the total increment $\inc(S,I,R) \le s$, using a \textit{dynamic} approach.
It creates a matrix $M$ where each cell $M_{i,j}$ has domain $\mathcal{P}(L) \cup \{\Box\}$.
$M_{i,j}$ represents the cardinality-maximal set with $sum(M_{i,j}) = i$ and $M_{i,j} \subseteq L_j$,
if $M_{i,j} = \Box$ it means that such subset does not exists.
When $M_{i,j} = \Box$ then, abusing of notation, $|M_{i,j}| = -1$.
The matrix is initialized with function \textit{Init} ~\ref{alg:init}.
This function will initialize just the first row and column.
It is trivial to see that $M_{0,j}$ will contain all the literals of $L_j$ 
with increment equal to 0, since the sum $i = 0$.
Thus, the first row will be 
$M_{0,j} = \{ \ell \mid \ell \in L_j \land inc(\ell,I,R) = 0\} \quad \forall j \in \{1, \hdots , n\}$.
When the sum $ i > 0$ then with $L_0 = \emptyset$ is impossible to reach $i$, 
that's why $M_{i,0} = \Box \quad \forall i \in \{1, \hdots , s\}$.
After initilization the algorithm ~\ref{alg:cmss} constructs each cell $M_{i,j}$
, with $ 1 \ge i \ge s$ and $1 \ge j \ge n$, starting from $M_{i,j-1}$ and $M_{i-w,j-1}$,
where $w = \inc(\ell_j,I,R)$.
Let $S_{\overline{\ell}}$ and $S_{\ell}$ 
represent the cardinality maximum sets, considering literals in $L_j$ and having 
a total increment equal to $i$, with \(\ell_j\) 
non-active and \(\ell_j\) active, respectively. 
By selecting the larger set between these two, 
we obtain the maximum subset considering literals in $L_j$.
$S_{\overline{\ell}} = M_{i,j-1}$, since is the largest set giving as sum $i$
and $\ell$ is not active given that $M_{i,j-1}$ considers just the first $j-1$ literals;
note, this does not mean that $M_{i,j-1} \cap \{\ell\} = \emptyset$, since can happen 
that for some $d < j$ $M_{i,d} = M_{l,k} \ext \ell_d$ and $\group(\ell_d) = \group(\ell_j)$.
Instead finding $S_{\ell}$ is not so trivial, we cannot just take $M_{i-w,j-1}$,
since $M_{i-w,j-1}$ could containt a literal $\ell_d$ such that $d < j$,
$\group(\ell_d) = \group(\ell_j)$ and in that case 
$\mathit{sum}(M_{i-w,j-1}) = \mathit{sum}(M_{i-w,j-1} \ext \ell_j)$;
it is due to the fact that $\ell_j$ would not be active.
Since $L$ is ordered, and within each group there is a descending order, and since each 
cell $M_{i,j}$ is computed from left to right
then $M_{i-w,j-1} \cap \lits_{\group(\ell)} = \emptyset$ is also a \textit{necessary}
condition to have $\ell$
active in $M_{i-w,j-1} \ext \ell$
(so it becomes necessary and sufficient).
Since, by construction, $M_{i-w,j-1}$ can contain just literals \textit{`greater'} (in terms of 
increment) of $\ell_j$, and if this happens also $\ell_j$ is inside $M_{i-w,j-1}$ then 
it is sufficient to check that $M_{i-w,j-1} \cap \{\ell_j\} = \emptyset$.
Taking $M_{i-w,k} \ext \ell_j$ with $k = \arg \max_{k \in \{1,\hdots,j-1\}} \ell_j \not\in M_{i-w,k}$ 
yields the cardinality maximum subset with a sum of \( i \), and \(\ell_j\) active.
Hence, $S_{\ell} = M_{i-w,k}$.
Subsequently, after the for loop is completed all cells of the matrix are correct 
and it is enough to take the largest set in the last column, that is, the 
largest set giving as sum a value less or equal to $s$ considering all literals in $L$.
\begin{algorithm}[h]\small
\caption{Cardinality Maximum Subset Sum (cmss)}
    \label{alg:cmss}
    \SetKwInOut{Input}{Input} 
    \SetKwInOut{Parameters}{Parameters}
    \SetKwInOut{Output}{Output}
    \SetKwBlock{Loop}{Loop}{end}
    \Input{L, $I : \text{interpretation}$, $s \in \mathbb{N}$}
    \Parameters{\(group : \text{Group Function}\), \(inc : \text{Increment Function}\)}
    \Output{$S$: Set of literals representing the maximum subset}
    \Begin{
        $n \gets |L|$\\
        $M \gets \mathit{Init}(I,R,L,s)$\\
        \For{\(i \in  \{ 1, \hdots, s\}\) }{
            \For{\(j \in  \{ 1, \hdots, s\}\) }{
                \(\ell \leftarrow L[j - 1]\)\\
                \(w \leftarrow inc(\ell)\)\\
                \(M_{i,j} \leftarrow M_{i,j-1}\)\\
                \If{\(i \geq w\)}{
                    \If{$(\Box \not\in M_{i,j}) \lor (\Box \not\in M_{i - w,j - 1})$}{
                        \For{$k \gets j-1$ \KwTo $0$}{
                            \If{\(\ell \notin M_{i - w,k}\) $\lor$ $\Box \in M{i - w,k}$}{
                                \textbf{break}
                            }
                        }
                    $\mathit{bool} \gets \Box \not\in M_{i - w,k} \land \ell \notin M_{i - w,k}$ \\
                    $wls \leftarrow M_{i - w,k} \ext \ell \quad \textbf{if } \mathit{bool}  \textbf{  Else  } \Box$\\
                    $M_{i,j} \leftarrow \arg \max \{|wls|, |M_{i,j - 1}| \}$
                }
            }
            }
        }
        $S = \arg\max_{i \in \{0, \hdots, s\}} |M_{i,n}|$ \\
        \Return{$S$}
    }
\end{algorithm}

\begin{algorithm}[h]\small
    \caption{Init}
        \label{alg:init}
        \SetKwInOut{Input}{Input}
        \Input{interpretation $I$, reason $R$, $L$, threshold $s \in \mathbb{N}$}.
        \SetKwBlock{Loop}{Loop}{end}
        \Begin{
            $M_{0,j} = \{ \ell \mid \ell \in L_j \land inc(\ell,I,R) = 0\} \quad \forall j \in \{1, \hdots , |L|\}$\\
            $M_{i,0} = \Box \quad \forall i \in \{1, \hdots , s\}$\\
        }
\end{algorithm}

\subparagraph{Proof Correcntess}
To formally prove that algorithm ~\ref{alg:cmss} is correct a lemma 
has to be proved.
\begin{lemma}
    Given a set of literals $L$, a literal $\ell \in L$ and a set $S_{\bar{\ell}} \subseteq  L$
    and $S_{\ell} \subseteq L$ and a sum $s$.
    Let $S_{\bar{\ell}}$ is the maximum cardinality subset giving $s$ as sum with $\ell$ being an \textit{non-active} literal
    and $S_{\ell}$ is the maximum cardinality subset with $\ell$ being an \textit{active literal} giving $s$  as sum.\\
    Let $ S = \arg\max_{X \in \{S_{\ell}, S_{\bar{\ell}}\}} |X| $. \\ 
    Then $S \subseteq L$ is a maximum cardinality subset that gives as sum $s$.
    \label{lemma1}
\end{lemma}

\begin{proof}
    Let's assume by contradiction that $S$ is not the maximum cardinality subset that gives as sum $s$.
    So there exists a set $S' \subseteq L$ such that $|S'| > |S|$.
    If $\ell \not\in A_{S'}$ then $|S'| > |S_{\bar{\ell}}|$, that is $S_{\bar{\ell}}$ 
    is not the maximum cardinality set 
    with $\ell$ being a \textit{non-active literal} giving as sum $s$, but this is a contradiction.
    So $\ell \in A_{S'}$ then $|S'| > |S_{\ell}|$, that is $S_{\ell}$ is not the maximum cardinality set 
    with $\ell$ being a \textit{active literal} giving as sum $s$, but this is a contradiction.
    So $\ell \in A_{S'}$ and $ \ell \not\in A_{S'}$ and this is a contradiction.
\end{proof}

\begin{theorem}
    The Cardinality Maximal Subset Sum algorithm returns the cardinality maximal subset $S$ where its sum 
    of increments of $S$ is less than $\mathit{bnd} - 1$
\end{theorem}

\begin{proof}
    
\end{proof}

