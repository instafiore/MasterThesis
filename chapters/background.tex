\chapter{Background}

This chapter defines all the background needed to explain our work,
trying to guide the reading through a clear and intuitive idea.
Initially the syntax and semantics of normal
program (section $~\ref{sec:bg-syntax_semantics}$) is showed,
primarly focusing on notions relevant for this thesis, for instance on some
extension such as the SUM constraints.
Then some specific cases of SUM constraints, the so-called ALO and AMO constraints, 
is explained in section $~\eqref{sec:bg-clauses_AMO}$.
Afterwards, the current state-of-the-art ASP solver algorithm to find a stable model 
is discussed (section $~\ref{sec:bg-SM}$), focusing on the concept of \textit{propagator}.

\section{Syntax and Semantics}
\label{sec:bg-syntax_semantics}

This section proceeds in a bottom-up fashion, introducing the 
most basic elements to the concept of \textit{program}
and \textit{stable model}.

The first element is the set of \textit{atoms}, let $\mathcal{A}$ be such set.
$\neg$ is a symbol representing the common negation in logic.
A $\emph{literal}$ is an atom with possibly the negation symbol in front of it.
For instance $a \in \mathcal{A}$ is a literal (and an atom) and $\neg b$ with $b \in \mathcal{A}$
is a literal (but not an atom); \textit{a} is said to be a \textit{positive literal} instead
$\neg b$ is a negative literal. In a more formal way: given a literal $\ell \in L$, $\ell$ is 
positive if $\ell \in \mathcal{A}$ otherwise it is negative. Let $L$ be a set of literals.
Given $\ell \in L$ then $\overline{\ell}$ denotes its complement, if $\ell$ is a positive
literal, i.e. $\ell = a \in \mathcal{A}$, then $\overline{\ell} = \neg a$ else 
when $\ell = \neg a$ (negative literal) with $a \in \mathcal{A}$ then $\overline{\ell} = a$.
A set of literal $L$ can be negated, written $\overline{L}$;
$\overline{L}$ is equivalent to the set of literals of $L$ where each literal is negated,
that is, $\overline{L} = \{ \overline{\ell} \mid \ell \in L \}$.

Each atom can be mapped to a truth value (boolean value) by an \textit{intepretation}.
An \textit{intepretation} (or \textit{assignment}) \textit{I} is a set of 
literals where $I \cap \overline{I} = \emptyset$.
If $\mathcal{A} \subseteq (I \cup \overline{I})$ then $I$ is called \textit{total-intepretation},
otherwise it is a \textit{partial-intepretation}.
On one hand if $\ell \in I$ then $\ell$ is \textit{true} under $I$, on the other hand 
if $\ell \in \overline{I}$ then it is \textit{false} under $I$.
If either $\ell \not\in I$ and $\ell \not\in \overline{I}$ then $\ell$ is said to be \textit{undefined}. 
Abusing of notation, If $\ell \in I$ let $I^{\top}(\ell) := 1$, 0 otherwise;
if $\ell \not\in \overline{I}$ let $I^{\neg\bot}(\ell) := 1$, 0 otherwise.

Now the first main brick can be presented: the rule.
A rule is a classic implication in propositional logic.
\begin{align}\label{eq:rule}
    r: \quad p \leftarrow \ell_1, \ldots, \ell_n
\end{align}
where \textit{p} is an atom and $\ell_1, \ldots, \ell_n$ with $n \ge 0$ are literals.
The rule r  $~\ref{eq:rule}$ is equivalent to $\ell_1 \land\, \ldots \, \land  \ell_n \rightarrow p$.
As in propositional logic $p$ and $\ell_1, \ldots, \ell_n$ are named respectively \textit{head}
and \textit{body} of the rule.
The head of \textit{r} is defined by the symbol $\mathit{H}(r) := p$
and the body by the set
$\mathit{B}(r) := \{\ell_1, \ldots, \ell_n \}$.
Each rule has a \textit{positive} and \textit{negative} part, and it is stricly 
linked with the concept of positive and negative literal.
On one hand, the positive body of the rule \textit{r}, named $\mathit{B}^+(r)$, is the set
of positive literals of \textit{r}, that is, 
$\mathit{B}^+(r) = \{ \ell \mid \{\ell\} \cap \mathcal{A} \ne \emptyset \}$.
On the other hand, the negative body of the rule \textit{r}, named $\mathit{B}^-(r)$, is the set
of negative literals of \textit{r}, namely, 
$\mathit{B}^-(r) = \{ \ell \mid \{\ell\} \cap \mathcal{A} = \emptyset \}$

Now the relation $\models$ (satisfies, or is model of) is inductly 
defined: let $I$ be an intepretation, if $\ell \in I$ then $I \models \ell$;
if $\ell \in \overline{I}$ then $I \models \overline{\ell}$;
if $I \models \ell_+$ for every $\ell_+ \in B^+(r)$ and 
$I \models \ell_-$ for every $\ell_-\in B^-(r)$, then $I \models B(r)$;
if whenever $I \models B(r)$ also $I \models H(r)$ then $I \models r$.
A (normal) program $\Pi$ is a set of rules.
$M(\Pi) = \{ I \mid I \models \Pi\}$ is the set of \textit{models} of $\Pi$. 

Let's now shift to another concept that allow us to 
transition from a normal program to an extension of it: the concept of \textit{SUM constraint}.
A SUM constraint (or simply constraint) has the following form
\begin{align}\label{eq:sum}
    \textsc{sum}\{w_1 : \ell_1;\ \cdots;\ w_n : \ell_n\} \geq b
\end{align}
where $n \ge 0$, $\{\ell_1, \hdots, \ell_n\}$ is a set of literals such that 
$\ell_i \ne \ell_j$ for all $i,j \in \{1, \hdots, n\}$ such that $i \ne j$
and $\{b, w_1, \hdots, w_n \}$ is a set of naturals numbers.
Let $\sigma$  be a constraint of the form $~\ref{eq:sum}$ then $\mathit{bnd}_{\sigma} = b$
represents the \textit{bound} of the constraint $\sigma$; $w_1, \hdots, w_n$
are the weights of each literal in $\sigma$; $lits_{\sigma}$ is the set of literals of $\sigma$.
Let's specify the relation $\in$ as $(w_i, \ell_i) \in \sigma$ to be read  
as $(w_i : \ell_i)$ is an element in (the aggregation set of) $\sigma$;
The function $\mathit{wh_{\sigma}}(\ell_i) = w_i$, namely, this is a function that 
maps every literal of $\sigma$ to its weight.
Intuitively, the constraint is satisfied w.r.t. an interpretation $I$ 
if summing the weight of the true literals under $I$ yields to value greater or equal to the bound.
More formally, extending the relation $\models$, $\sigma$ is satisfied if 
$\sum_{i=1}^{n} \mathit{wh}_{\sigma}(l_i) \cdot I^{\top}(l_i) \ge \mathit{bnd}_{\sigma}$,
written $I \models \sigma$.
Note that $\sigma$ may be omitted from the above notation if its meaning is clear from context.\\
Just as a side note: in ASP-Core-2 standard
~\cite{DBLP:journals/tplp/CalimeriFGIKKLM20} 
the constraint  $~\ref{eq:sum}$ is written as the headless rule
$\mathit{:-} \quad \#sum\{w_1,\ell_1 : \ell_1; ...; w_n,\ell_n : \ell_n\} < b.$
Now a more formal definition of program can be given:
a program $\Pi$ is a set of rules and constraint, referred as $\mathit{rules}(\Pi)$ 
and $\mathit{constraints}(\Pi)$ respectively.
The sets of rules and constraints define the set of atoms of the program and they are named $\mathit{atoms}(\Pi)$.
Finally, $I$ satisfies $\Pi$, written $I \models \Pi$, if $I \models r$ for all $r \in \mathit{rules}(\Pi)$
and $I \models \sigma$ for all $\sigma \in \mathit{constraints}(\Pi)$.

To make the discussion made so far more understandable, we will introduce the following example:
\begin{example}[Running example]
    \label{ex:running}
    Let $\Pi_{\mathit{run}}$ be the following:
    \begin{align*}
        \begin{array}{rll}
            r_{\alpha\phantom{'}}: & \alpha\phantom{'} \leftarrow \neg \alpha' & \alpha \in \{x,y,z\}\\
            r_{\alpha'}: & \alpha' \leftarrow \neg \alpha & \alpha \in \{x,y,z\}\\ 
            \sigma_{1}: & \textsc{sum}\{
                1 : \overline{x};\ 1 : \overline{y}\phantom{;\ 2 : z}
            \} \geq 1 \\
            \sigma_{2}: & \textsc{sum}\{
                1 : x;\ 2 : y;\ 2 : z
            \} \geq 3
        \end{array}
    \end{align*}
Note that there are six atoms ($x,y,z,x',y',z'$), six rules and two constraints.
$\mathit{wh}_{\sigma_1}(\overline{x}) = 1, \mathit{wh}_{\sigma_1}(\overline{y}) = 1,
\mathit{wh}_{\sigma_2}(x) = 1,\mathit{wh}_{\sigma_2}(y) = 2,
\mathit{wh}_{\sigma_2}(z) = 2, \mathit{bnd}_{\sigma_1}=1,\mathit{bnd}_{\sigma_2}=3$.
\end{example}
A possible (total) interpretation $I$ satisfying $\Pi_{\mathit{run}}$ is:
$I_1 = \{ x, z, \overline{y}, x', y', z' \}$.
Since, $\sum_{i=1}^{2} \mathit{wh}_{\sigma_1}(\ell_i) \cdot I_1^{\top}(\ell_i) = 
\sum 1 \cdot I_1^{\top}(\overline{x}) + 1 \cdot I_1^{\top}(\overline{y}) = 1 \ge \mathit{bnd}_{\sigma_1}$
and $1 \cdot I_1(x) + 2 \cdot I_1^{\top}(y) + 2 \cdot I_1^{\top}(z) = 3 \ge \mathit{bnd}_{\sigma_2}$.
Given a program $\Pi$ and an intepretation $I$, its \textit{reduct} is
defined as follows: $\Pi^{I} = \{ H(r) \leftarrow B^+(r) \mid r \in \mathit{rules}, I \models B(r)\}$,
please note that $\mathit{constraints}(\Pi^{I}) = \emptyset$.
One interpretation may differ from another due to its stability.
An interpretation $I$ is a stable model of a program $\Pi$ if $I \models \Pi$ and there is 
no $J \subset I$ such that $J \models \Pi^I$ .
Let $\mathit{SM}(\Pi)$ denote the set of stable models of $\Pi$.
Taking in consideration example $~\ref{ex:running}$ we can notice that 
$I_1$ is not a stable model, that is, $I_1 \not\in \mathit{SM}(\Pi_{\mathit{run}})$.
This is because taking the partial assignment $J_1 = \{y'\}$
then $ J_1  \models \Pi_{\mathit{run}}^{I_1}$ since $\Pi_{\mathit{run}}^{I_1} = \{ y' \leftarrow  \}$
and $J_1 \subset I_1$.
Instead $I_2 = \{ x, z, \overline{y}, \overline{x'}, y', \overline{z'} \} \in \mathit{SM}(\Pi_{\mathit{run}})$.

Continuing talking about the above example, a last consideration that let us to move towards the next 
chapter has to be addressed.
The constraint $\sigma_1$ is a special one, it says: at least 1 of the 2 literals
has to be satisfied. Notice that all the literals are flipped, 
so referring to the literals without being flipped it is 
actually saying `at least 1 of the 2 have to be falsified' , and since $2 = 1 - 1$ it can be 
reformulated as 'at most 1 of them can be satisfied'.
Thus, with an \textit{At least One (ALO)} sum constraint is possible to define an \textit{At Most One (AMO)}
constraint.

\section{ALO (clauses) and AMO as a Special Case}
\label{sec:bg-clauses_AMO}

Given a set $\{\ell_1, \hdots, \ell_n\}$, with $n \ge 1$, an \textit{At Least One (ALO)} 
constraint over this set will enforce to have at least one literal true.
As in the example $~\ref{ex:running}$ it can be expressed as:
\begin{align}\label{eq:ALO}
    \textsc{sum}\{1 : \ell_1;\ \cdots;\ 1 : \ell_n\} \geq 1
\end{align}
This constraint can be written as a \textit{set} of the form 
\begin{align}
    \label{eq:clauses}
    \{\ell_1, \hdots, \ell_n\}
\end{align} 
Usually this constraint is named also \textit{clause} (following the \textit{CNF} notation 
of the propositional logic).
Modern ASP solvers enrich the input program with clauses enforcing $I$ to be a 
model of rules of the form \eqref{eq:rule} (i.e., $\{p, \overline{\ell_1}, \ldots, \overline{\ell_n}\}$).
If over this set instead is enforced an \textit{At Most Constraint}, i.e., at most 
one literal is true, the following constraint is introduced:
\begin{align}\label{eq:amo}
    \textsc{sum}\{1 : \overline{\ell_1};\ \cdots;\ 1 : \overline{\ell_n}\} \geq n - 1
\end{align}
To enforce that at most
one literal is true of a given set it is enough to enforce that 
at least $n-1$ literals are falsified.
Since, intuitively, if two different literals are satisfied then $n-2$ literals are not falsified.
More formally given an interpretation $I$, $I$ satisfies at most 1 literal if 
$\sum_{i = 1}^{n}{I^{\top}(\overline{\ell_i})} \geq n - 1$, or equivalently
$\sum_{i = 1}^{n}{I^{\top}(\ell_i)} \leq 1$.
The AMO constraint \eqref{eq:amo} is compactly written $\textsc{amo}\{\ell_1,\ldots,\ell_n\}$.
\begin{example}[Continuing Example~\ref{ex:running}]
    Note that $\sigma_1$ is the clause $\{\overline{x}, \overline{y}\}$, or also the AMO constraint $\textsc{amo}\{x, y\}$.
\end{example}
\section{Stable Model Search, Propagators and Learning}
\label{sec:bg-SM}

Currently ASP solvers employ a \textit{conflict-driven clause learning} (CDCL) algorithm
\cite{DBLP:journals/ai/GebserKS12} to search a stable model.
This is an algorithm also used in a lot of SAT solvers and it has been revealed to be a 
very effective one.
The CDCL follows a pattern called \textit{choose-propagate-learn}, where: 
the \textit{choose} phase consists in deciding a literal to become true, such literal 
is named \textit{branching literal};
\textit{propagate} phase involves to deterministically derive new consequences from the current
state (interpretation); \textit{learn} phase, when a conflict arises, understands
some new \textit{clause} (constraint) that was not explicity defined in the program.
To dive into each of these phases, understanding the CDCL, some previous concepts have
to be mentioned.

A \textit{conflict} occurrs in an intepretation $I$,
when two literals $\ell, \overline{\ell}$ are together in $I$.
Given two clauses $C,D$ of the form \eqref{eq:clauses} (as described in $~\ref{sec:bg-clauses_AMO}$) and a literal 
$\ell$ such that $\ell \in C$ and $\overline{\ell} \in D$ then the \textit{resolution}
(~\cite{davis1960computing}, ~\cite{robinson1965machine})
step of $C$ and $D$ upon $\ell$, written $C \otimes_{\ell} D$, is equal to 
$(C \setminus \{\ell\}) \cup (D \setminus \{\overline{\ell}\})$.
Intuitively, if $C \in \Pi$ and $D \in \Pi$ 
then since an intepretation cannot simultaneously
satisfying $\ell$ and $\overline{\ell}$ then $C \setminus \{\ell\}$ or 
$D \setminus \{ \overline{\ell}\}$ have to be satisfied, thus the following clause 
$(C \setminus \{\ell\}) \cup (D \setminus \overline{\ell})$ has to be satisfied.
Please note that if an intepretation $I \models \Pi$ then $I \models C \otimes_{\ell} D$.
So $C \otimes_{\ell} D$ is implied by the program,
written $\Pi \models C \otimes_{\ell} D$.
A clause $C$ is \textit{redundant} in $\Pi$ if $\Pi \models C$.
Given a clause $C$ then an assignment $I$ that \textit{blocks}
$C$ is an assignment that falsifies all literals (and no others)
of $C$, i.e., $I = \overline{C}$. 

In the \textit{choose} phase a branching (undefined) literal is selected.
Every branching (or decision) literal is paired with a \textit{decision level}.
To understand the decision level is enough to 
now that at each choose phase the corresponding literal is added into a 'list' and
the decision level is the index into the list (index starting from 1). 
The term `\textit{backjumping} to a certain level $l$' refers to the process of `forgetting' 
every decision made after decision level $l$, 
including the respective propagated literals, and then 
resuming the decision-making process starting from level $l$.


In the \textit{propagate}, given an intepretation $I$,
potentially some literal $\ell$ is derivated 
(or propagated) using a \textit{Propagate} function according to some \textit{inference rule}.
An inference rule is a logical construct used  
to derive new \textit{literals} (\textit{conclusions}) 
from a current intepretation $I$ (\textit{premises}). 
It specifies the conditions under which certain statements (the conclusions) 
can be inferred from other statements (the premises).
After a literal $\ell$ is inferred, thanks to some inference rule, then a \textit{reason}, written $\mathit{reason}(\ell)$, 
is specified. This reason defines \textit{why} $\ell$ has been propagated.
More in detail, the reason is a clause of this form 
\begin{align}
    \label{eq:reason_form}
    R = \{ \ell \} \cup \{\overline{\ell_1},\hdots, \overline{\ell_n}\}
\end{align}
\eqref{eq:reason_form} represents the implication 
rule $\ell_1 \land \hdots \land \ell_n \rightarrow \ell$ and it specifies 
that when all the literals $\ell_1 \land \hdots \land \ell_n$ are true then also 
$\ell$ must be true. $B(R) = \overline{\{\overline{\ell_1},\hdots, \overline{\ell_n}\}}$
is the body of the reason and $H(R) = \{\ell\}$.
$R$ is said to be a reason of the literal $z$ if $H(R) = \{z\}$.
$J_R= B(R)$ is the assigment satisfying the body of $R$; note that $J_R$ is blocking $R \setminus H(R)$.
A clause $C$ of the form \eqref{eq:reason_form} is a reason for $\ell$, under the assignment $I$, 
if $I \supseteq B(C)$ and under intepretation $J_c$ the propagator 
infers  $\ell$, thanks to some inference rule.
Finally, $\mathit{reason}(\ell)$ has to be a reason of $\ell$ under $I$.
Can happen that inferring a literal $\ell$ creates a conflic (i.e., $\overline{\ell} \in I$),
in that case $\ell$ is a \textit{conflict literal}.
Moreover $\ell$ is also paired with a decision literal 
that it is inherited from the current decision level of the search.
\textit{Unit Propagation} is a specific kind of propagation, it is applied when given a clause 
$C$ and a literal $\ell \in C $, $(C \setminus \{\ell\}) \cap \overline{I} = C \setminus \{\ell\}$.
In this case the only way to satisfy $C$ is setting $\ell$ to true, thus $\ell$
is propagated to true; the reason is exactly because the other literals are falses, so 
$reason(\ell) = C$.
Please not the $reason(\ell)$ respresents $\overline{\ell_1} \land \hdots \land \overline{\ell_n} \rightarrow \ell$, where 
$\{\ell_1 \land \hdots \land \ell_n\} = C \cap \overline{I}$.The \textit{Propagate} function is implemented using multiple \textit{propagators},
calling them sequentially according to a priority list.

The whole algorithm, initially starts with an empty assignment (all literals undefined)
and a the decision level (\textit{dl}) to 0  ,
then the \textit{propagate} phases takes place, inferring all the consequences.
If a conflict is detected it means that the program is not satisfiable, since 
without any choice we got a conflict.
Else, if no conflict is detected then the \textit{choose} phase decides the new branching literal.
An important note is that just the a branching literal $\ell$ has a reason of the form $\{\ell\}$, representing
$\rightarrow \ell$ ($\ell$ must be true), since it 
does not follows from any logic reasoning.
Then, again the propagate phase is executed.
If a conflict is detected then a process named \textit{conflict analysis} starts.
Starting from the reason of the conflict literal a (\textit{backward}) resolution upon a literal of the last
decision level is perfomed. This step is iteratively done until the new obtained (redundant) clause 
contains just one literal of the current decision level, this clause in called
\textit{Unique Implication Point (UIP)}.
Given the UIP then a \textit{backjump} operation is perfomed to the second highest decision level
(\textit{assertion level}); to update the internal state following a backjump, the \textit{Unroll}
function is invoked for each propagator.
The assertion level is special in the sense that it is the deepest level
at which adding the conflict-driven clause (UIP) would allow unit propagation to derive
a new implication using that clause. Since the literal with highest dl after backjumping 
is the only undefined literal inside UIP then the unit propagation will infer that literal,
that is, will flip the previous value.
When the highest literal in the UIP has 0 as dl then the assertion level is -1 by default.
Some \textbf{important} final notes: the reason for a literal $\ell$
must include at least one literal from the most recent decision level, excluding $\ell$
itself, otherwise, unit propagation 
cannot be performed; the smaller the cardinality of the conflict literal's reason, 
the higher the potential jump in the search space.
After this, a new choice is made until either all atoms are defined or the
assertion level is -1.
The above described algorithm is defined below and it is mainly based on the algorithm 
defined in \cite{DBLP:series/faia/SilvaLM09}.
\begin{algorithm}[h]\small
    \caption{Typical CDCL algorithm}
        \label{alg:cdcl}
        \SetKwInOut{Input}{Input}
        \Input{An ASP program $\Pi$}
        \SetKwBlock{Loop}{Loop}{end}
        \Begin{
            $I \gets [\,]$\\
            $dl \gets 0$\\
            \If {(\textsc{Propagate}$(\Pi, I) == \text{CONFLICT}$)}{
                \Return{Unsat}
            }
            \While{\textsc{not} \textsc{AllVariablesAssigned}$(\Pi, I)$}{
                $\ell = \textsc{PickBranchingLiteral}(\Pi, I)$ \\
                $dl \gets dl + 1$ \\
                $\mathit{store}(I,\ell,dl)$\\
                \If {(\textsc{Propagate}$(\Pi, I) == \text{CONFLICT}$)}{
                    $\beta = \textsc{ConflictAnalysis}(\Pi, I)$\\         
                    \If {($\beta < 0$)}{
                        \Return{Unsat}
                    }
                    \Else{
                        \textsc{Backjump}$(\Pi, I, \beta)$\\
                        $dl \gets \beta$ \\
                    }
                }
            }
        \Return{I}
        }
        
\end{algorithm}

Let's assume for now that the Propagate function is exactly equal to the propagator
implementing \textit{Unit propagation},
this is usefull for the next example.

\begin{example}\label{ex:unit-propagation}
    Let $\Pi$ have, among others, the rules
    \begin{align*}
        \begin{array}{ccc}
            x \leftarrow \neg z \qquad &
            y \leftarrow \neg z \qquad &
            w \leftarrow x, y
        \end{array}
    \end{align*}
    Hence, a modern ASP solver materializes the clauses
    \begin{align*}
        \begin{array}{ccc}
            \{x,z\} \qquad &
            \{y,z\} \qquad &
            \{w,\overline{x},\overline{y}\}
        \end{array}
    \end{align*}
    For readability at each literal is associated the relative decision level 
    as superscript.
    Let's assume that the current interpretation is 
    $I = [\overline{w}^1]$ and the decision level is $dl= 1$.
    Since no propagation is performed the algorithm goes 
    directly to the propagate phase, let's assume that 
    $\overline{z}$ is selected.
    Then, by unit propagation, $x$ and $y$ are inferred,
    thanks to the first two clauses.
    Unit propagation infers $x, y$ from the first two clauses, and then $y$ 
    (a conflict literal) from the third clause.
    The correspective reasons are: $reason(x) = \{ x, z \}$,
    $reason(y) = \{ y, z \}$ and $reason(\overline{y}) = \{w,\overline{x},\overline{y}\}$.
    The current intepretation $I$ is equal to
    $[\overline{w}^1,\overline{z}^2, y^2, x^2]$.
    Since there are more than 1 literal of the decision level 2 (the last one) in the conflict clause
    ($\{w,\overline{x},\overline{y}\}$)
    a backward resolution step is perfomed.
    Let's take arbitrarily $\overline{x}$:
    $\{w,\overline{x},\overline{y}\} \otimes_{\overline{x}} \{ x, z \}= \{w, \overline{y}, z\}$.
    Since both $\overline{y}, z$ are with decision level 2 then let's take arbitrarily $\overline{y}$:
    $\{w, \overline{y}, z\} \otimes_{\overline{y}} \{y,z\} = \{w, z, z\} = 
    \{w, z\}$.\\
    $\{w, z\}$ is \textit{UIP} so let's backjump to the assertion level 1.
    This will unit propagate $z$ with $reason(z) = \{w, z\}$.
    So the intepretation will look like: $I = [\overline{w}^1, z^1]$
\end{example}

The main difference between a SAT solver and a ASP solver can be summarized focusing on 
the function $\mathit{Propagate}(\Pi,I)$.
SAT solvers employ mainly unit propagation inside the \textit{Propagate} function,
instead ASP solvers use, in addition on unit propagation, other two fundamental  
propagation functions: \textit{Unfounded-free propagation} and \textit{ConstraintPropagation}.
The first one (\textit{Unfounded-free propagation}) assesses that the intepretation is a stable model, but its details
are out of the scope of this work.
The second one (\textit{ConstraintPropagation}) is used to derive new consequences from 
the constraints present in the program. 
For a SUM constraint $\sigma$ of the form \eqref{eq:sum},
solvers typically employ a 
specific \textit{constraint propagator}
~\cite{DBLP:conf/iclp/GebserKKS09,DBLP:journals/fuin/FaberLMR11} leveraging the concept of 
\textit{max possible sum (mps)}.
Given an intepretation $I$, a literal $\ell$ and a constraint $\sigma$ of the form \eqref{eq:sum}, 
a max possible sum, considering $\ell$ as false, is $\mathit{mps}_{\sigma}(I,\ell) = 
\sum_{j \in [1..n], \ell_j \neq \ell}{w_j \cdot I^{\neg\bot}(\ell_j)}$;
intuitively, if all the undefined literal were true and $\ell$ was false
then the sum would be equal to $\mathit{mps}_{\sigma}(I,\ell)$.
An inference rule of a sum constraint essentially adds to $I$ the literal $\ell$ 
if $\ell$ is required to (possibly) reach the bound $b$, i.e., if
\begin{align}\label{eq:sum:propagation}
    \mathit{mps}_{\sigma}(I,\ell) < \mathit{bnd}_{\sigma}
\end{align}
In this case \textit{reason}$(\ell) = \{\ell\} \cup lits_{\sigma} \cap \overline{I}$.
A rational behind this is as follows: if in the best case (that is, when the sum is $\mathit{mps}(I,\ell)$)
the sum does not reach the bound it means that $\ell$ cannot be false, that is, it is required to be true.
Intuitively, the falses literals are the only justification why the bound $\mathit{bnd}_{\sigma}$
could be not reached if $\ell$ was not added to $I$.
In the special case of \eqref{eq:amo}, that is, if $\sigma$ is $\textsc{amo}\{\ell_1, \ldots, \ell_n\}$, the literal $\overline{\ell}_i$ ($i \in [1..n]$) is added to $I$ if there is $\ell_j$, with $j \neq i$, such that $\ell_j \in I$.
In this case, $\mathit{reason}(\overline{\ell_i})$ is $\{\overline{\ell_i}, \overline{\ell_j}\}$.

\begin{example}[Continuing Example ~\ref{ex:running}] \label{ex:sumpropagator}
    If $I$ is empty, no literal can be inferred from $\sigma_1$ and $\sigma_2$.
    If $I$ is $[\overline{z}]$, then the application of \eqref{eq:sum:propagation} to the literals of $\sigma_2$ gives
    \begin{align*}
        2 \cdot [\overline{z}]^\uparrow(y) + 2 \cdot [\overline{z}]^\uparrow(z) = 2 \cdot 1 + 2 \cdot 0 = 2 < 3\\
        1 \cdot [\overline{z}]^\uparrow(x) + 2 \cdot [\overline{z}]^\uparrow(z) = 1\cdot 1 + 2 \cdot 0 = 1 < 3\\
        1 \cdot [\overline{z}]^\uparrow(x) + 2 \cdot [\overline{z}]^\uparrow(y) = 1 \cdot 1 + 2 \cdot 1 = 3 \not< 3
    \end{align*}
    Hence, $x$ and $y$ are inferred with
    $\mathit{reason}(x) = \{x,z\}$ and
    $\mathit{reason}(y) = \{y,z\}$.
    Note that, once $I = [\overline{z},x,y]$, the application of \eqref{eq:sum:propagation} to $\sigma_1$ gives
    \begin{align*}
        1 \cdot [\overline{z},x,y]^\uparrow(\overline{y}) = 1 \cdot 0 = 0 < 1\\
        1 \cdot [\overline{z},x,y]^\uparrow(\overline{x}) = 1 \cdot 0 = 0 < 1
    \end{align*}
    Therefore, a conflict is raised, say because $\overline{y}$ (or similarly $\overline{x}$) 
    is added to $I$ with $\mathit{reason}(\overline{y}) = \{\overline{x}, \overline{y}\}$.
\end{example}
